\documentclass{article}

\usepackage[tex4ht]{hyperref}
\hypersetup{colorlinks=false}

\begin{document}



\parindent=0pt
\addtolength{\parskip}{4pt}
\addtolength{\textwidth}{10em}
\addtolength{\textheight}{12em}
\def\tc{\mathaccent"7017}


\centerline{\large{\bf Curriculum vitae of Tatyana Shaposhnikova}}

\noindent
Part time job: Dept. of Mathematics, Royal Institute of Technology\\
Address: Institutionen f\"or mathematik, KTH, 10044 Stockholm, Sweden\\
Email: \href{mailto:tat@math.kth.se}{tat@math.kth.se}\\
\url{http://people.kth.se/~tat}


\noindent{\bf Education}
\begin{itemize}
         \item Graduated from the Leningrad University in 1969
         \item 1969--1972 postgraduate student of the same University
         \item Candidate of Science (Ph. D.) Degree in 1973 from the
Leningrad University
\end{itemize}

    \noindent{\bf Employment}
\begin{itemize}
         \item 1973--1979  assistant professor, Department of
Mathematics, Leningrad
          Military Academy
         \item 1979--1983  assistant professor, Department of
Mathematics, Leningrad
          Institute of Refrigeration Industry
          \item 1983--July 1990  associate professor, docent,
Department of Mathematics, Leningrad Finance
         and Economics Institute
          \item  July, 1991-- September, 2013   associate professor
(universitetslektor), Department of Mathematics, the
         University of Link\"oping
         \item 2004--2008, each year January-June -- full professor, Department of Mathematics, Ohio State University
        \item October 2013 -- part time job,  Department of Mathematics, Royal Institute of Technology, Stockholm

\end{itemize}

\noindent{\bf Awards}
\begin{itemize}
         \item March 2003, Verdaguer Prize of the French Academy of Sciences
         \item May 2010, Thur\'eus Prize of the Royal Society of Sciences in Uppsala
\end{itemize}


\noindent{\bf Books}

\begin{itemize}
         \item Maz'ya V., Shaposhnikova T. {\it Theory of Multipliers
in Spaces of
               Differentiable Functions}, Pitman, 1985 (Russian
               version: Leningrad University Press, 1986).
         \item Shaposhnikova T., Polyshchuk E. {\it Jacques Hadamard}, Nauka,
               Leningrad, 1990.
         \item Maz'ya V., Shaposhnikova T. , {\it Jacques Hadamard, a Universal
               Mathematician}, American Mathematical Society and
               London Mathematical Society, 1998.
         \item Maz'ya, V., Shaposhnikova, T.,
{\it Jacques Hadamard, un Math\'ematicien Universel}. EDP Sciences, Paris, 2005
(revised and extended translation from English).
         \item Kresin G., Maz'ya, V. {\it Sharp Real-Part Theorems. A Unified Approach}, Translated from Russian and edited by T. Shaposhnikova, Springer, 2007.
         \item Maz'ya V., Shaposhnikova T., {\it Jacques Hadamard, Legend of  Mathematics}. MCNMO Publishers, Moscow, 2008 (revised,  extended, and  authorized translation from English to Russian).
\item Maz'ya V., Shaposhnikova T., {\it Theory of Sobolev Multipliers with Applications to Differential and Integral Operators}, Grundlehren der Mathematischen Wissenschaften, vol. 337, Springer, 2009.
\item Maz'ya V.,  Soloviev, A., {\it Boundary Integral Equations on Contours with Peaks}, Translated from Russian and edited by T. Shaposhnikova,
Operator Theory Advances and Applications, vol. 196, Birkh\"auser, 2010.



\end{itemize}


    \noindent{\bf Selected symposium talks}
\begin{itemize}
         \item 1990 (October)--Meeting of the Swedish Mathematical
Society, Link\"oping
         \item 1991 (August)--International Conference on Potential
Theory, Amersfoort

         \item 1992 (March)--International Conference on Boundary
Value Problems, Rostock
          \item 1993 (January)--Conference on Sobolev Spaces and Their
Related Fields, Kyoto
         \item 1994 (June)--International Conference on Applied and
Induistrial Mathematics, Link\"oping
         \item 1995 (June)--IABEM Symposium on Boundary Element
Methods for Nonlinear Problems, Siena
          \item 1996 (May)--Workshop on Conformal Geometry, Trondheim
         \item 1996 (October)--Conference on Analysis, Numerics and
Applications of Differential and Integral Equations, Stuttgart
         \item 1998 (May) IABEM Symposium on Boundary Element Methods, Ecole
           Polytechnique, Palaiseau (Paris)
         \item 1998 (August)--A conference in honour of Vladimir Maz'ya,
           Functional Analysis, Partial Differential Equations and
           Applications, Rostock
         \item 1998 (October)--International Symposium in Memory of G.
Fichera ``Problemi Attuali dell' Analisi e della Fisica
               Matematica'', Taormina.
         \item 1999 (March)--Conference in Memory of S. Pr\"ossdorf
``Recent Advances in Analytical and Numerical Treatment of
               Operator Equations '', Chemnitz.
         \item 1999 (May)-- ``Braude College Days of Differential
Equations and Applied Analysis'', Karmiel.
         \item 1999 (June)--Conference ``Integral Inequalities and
Applications'', Cortona.
         \item 1999 (August)--A conference in honour of Lars
G{\aa}rding on Analysis and Mathematical Phisics , Lund
         \item 2000 (July)--IABEM 2000, Brescia.
         \item 2000 (July)--PDE 2000, Clausthal.
         \item 2000 (August)--A conference in honour of Jaak Peetre,
Function spaces, Interpolation Theory and related topics, Lund.
         \item 2000 (September)--IWOTA Portugal 2000, Faro.
         \item 2001 (June)--A conference in honour of Hans Triebel,
Function Spaces, Differential Operators and Nonlinear Analysis,
          Teistungen
         \item 2001 (July)--Workshop in Nonlinear Differential
Equations, Bergamo
         \item 2001 (August)--Third International ISAAC Congress, Berlin
        \item 2002 (June) IABEM Conference on  Acoustics, Mechanics, and the Related Topics of Mathematical Analysis, Frejus, France
         \item 2002 (August)-- invited talk at the Conference on
Nonlinear Analysis (Satellite to ICM 2002), Taipei
         \item 2002 (August)-- International Congress of
Mathematicians, Beijing
         \item 2003 (June)-- Workshop in Operator Theory in Analysis and
Engineering, Darmstadt
         \item 2004 (September) -- International Conference on Boundary Integral Methods: Theory and Applications, Reading
         \item 2005 (July) -- Conference on Operator Theory, Function Spaces and Applications, Aveiro
         \item 2006 (April) -- 1015th Meeting of the American Mathematical Society, Miami
         \item 2006 (June) -- Workshop on Function Spaces and Differential Equations, St. Catharines
         \item 2006 (August) -- Conference on Geometric Methods in Nonlinear PDE and Free Boundary Problems, St. Petersburg
         \item 2006 (September) -- invited talk at the International Conference on Complex Analysis and Potential theory (Satellite to ICM 2006), Gebze
         \item 2007 (April) -- 1027th Meeting of the American Mathematical Society, Tucson, Arizona
         \item 2008 (April) -- invited talk at the International Conference on Analysis, Operator Theory and Applications, Cancun
         \item 2008 (June) -- Conference on Analysis, PDEs and Applications on the occasion of the 70th birthday of Vladimir Maz'ya, Rome
        \item 2008 (August) -- Nordic--Russian Symposium in honour of Vladimir Maz'ya on the occasion of his 70th birthday, Stockholm
         \item 2009 (July) -- Seventh ISAAC Congress, London
       \item 2009 (July 29 - August 2)  -- Spectral Theory and Geometric Analysis, Boston
       \item 2009 (August 7 --9) -- Advances in Boundary Integral Equations and Related Topics, Newark, Delaware
       \item 2010 (June) -- Workshop on Harmonic Analysis and Related Topics, Lisbon
       \item 2010 (October) -- Workshop on Harmonic Analysis and Elliptic PDEs, Link\"oping
        \item 2011 (May) --Workshop on New Function Spaces in PDEs and Harmonic Analysis, Naples
        \item 2011 (June) Workshop on Geometric Properties of Parabolic and Elliptic PDEs, Cortona
        \item 2013 (December) -- Analysis of Partial Differential Equations, Symposium in honour of Professor Vladimir Maz'ya, on the occasion of his 75th birthday, Liverpool
        \item 2014 (September) -- Advances in Nonlinear PDEs, Conference in honor of Nina N. Uraltseva and on the occasion of her 80th anniversary, St. Petersburg
\end{itemize}

\bigskip

\noindent{\bf Research Grants}
    \begin{itemize}
         \item 1992/1993 and 1993/1994 academic years-- grants from
the Department of Mathematics, Link\"oping University
          \item 1999--2001 grant from the Swedish National Science
Research Coucil (NFR)
         \item 2002 grant from the Department of Mathematics,
Link\"oping University
         \item 2003-2005 grant from the Swedish National Science
Research Coucil (VR)
                 \item 2008-2010 grant from the Department of Mathematics,
Link\"oping University
     \end{itemize}


 \noindent{\bf Member of Editorial Boards}
   \begin{itemize}
     \item Complex Variables and Elliptic Equations. An International Journal, USA.
     \item Eurasian Mathematical Journal, Kazakhtan
     \end{itemize}

 \noindent{\bf Member of Committees}
  \begin{itemize}
     \item Ethics Committee  of the European Mathematical Society
   \end{itemize}

\noindent{\bf Lecture Courses}
\begin{itemize}
         \item 1973--1979 Leningrad Military Academy: calculus, linear
algebra, transform theory
         \item 1979--1983 Leningrad Institute of Refregeration
Industry: calculus, linear algebra, probability and statistics
         \item 1983--1990 Leningrad Finance and Economics Institute:
calculus, linear algebra, transform theory, elementary course
         in mathematical physic
         \item 1987 (April) University of Rostock: Function spaces and
theory of multipliers
         \item 1990-- Link\"oping University: calculus, linear
algebra, second cours calculus
         \item 2000 (March) University of Basilicata: Introduction to
the theory of Sobolev multipliers
         \item 2001 (January-February) University of Stuttgart:
Sobolev spaces and multipliers
        \item 2004-2008 (January -June) Ohio State University: Calculus and Analytic Geometry
\end{itemize}


\noindent{\bf Additional information}
\begin{itemize}
         \item 1987 (April) Visiting Professor, Rostock University
         \item 1991 (January) Research guest of Mittag-Leffler Institute
         \item 2000 (March) Visiting Professor, University of Basilicata
         \item 2001 (January-February) Visiting Professor, University
of Stuttgart
         \item 2003 (January-June) Visiting Professor,
Northeastern University
\item 1978--1990 Reviewer for Russian Mathematical Reviews
         \item 1983--1997 Reviewer for Zentralblatt
\end{itemize}



\end{document}
